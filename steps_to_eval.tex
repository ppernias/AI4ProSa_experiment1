\documentclass[11pt,a4paper]{article}

\usepackage[utf8]{inputenc}
\usepackage[T1]{fontenc}
\usepackage{lmodern}
\usepackage{geometry}
\usepackage{setspace}
\usepackage{enumitem}

\geometry{margin=2.5cm}
\onehalfspacing

\title{Pasos seguidos para la evaluación}
\date{2026-02-07}

\begin{document}
\maketitle



Todos los archivos han sido pasado a texto y los resultados de los formularios a csv, más fácil de cargar y menor peso.
\begin{enumerate}
    \item Cargar el diseño experimental .txt modificado (se adjunta)
     Ese es el diseño experimental, y ahora nos vamos a centrar en la sección sobre el test de contenidos que trata de medir la eficacia, es el instrumento 1. Simplemente dime si lo entiendes y que se busca, nada más.
    
    \item Haz un análisis tomando como referencia esa sección del diseño experimental y para ello vamos a pasarte los resultados de los test realizados por cuatro grupos diferentes de estudiantes que además como podrás observar está en 3 idiomas diferentes: grupo1 español, grupo2 alemán, grupo 3 alemán y grupo 4 checo. Cada fichero corresponde a cada uno de los grupos. Este análisis debe ser breve y debe incluir conclusiones para cada grupo y unas conclusiones finales sobre esta parte del experimento. Muy breve y directo. 
    
    \item Ahora te pasamos los resultados del test cuantitativo de alumnos, al igual que antes ha sido realizado por cuatro grupos. Haz el análisis con la sección correspondiente del diseño experimental, recuerda que mide la personalización y autonomía. Este análisis debe ser breve y debe incluir conclusiones para cada grupo y unas conclusiones finales sobre esta parte del experimento. Muy breve y directo. 

    \item Pasamos al siguiente instrumento, hace falta analizar el cuestionario rellenado por el docente y que corresponde a una sección específica del diseño experimental. En este caso solo disponemos los datos para el grupo1, grupo 2 y 3.  Análisis muy breve y directo. 

    \item Tengo las transcripciones de las entrevistas pero solo para los grupos de alemán, analízalas según los objetivos planteados en el diseño experimental.

    \item Perfecto, formula las conclusiones finales del experimento unificando los cuatro instrumentos. Por favor sé claro, conciso y directo, no quiero narrativas extensas.

    \item Con lo visto te paso las notas tomadas sobre la evaluación he realizado algunas modificaciones, realiza la traducción al inglés británico y dale formato de Latex para Overleaf.
    
    \item Preparar una versión resumida para informe de proyecto o deliverable europeo, en inglés Británico y en Latex por favor


\end{enumerate}

\end{document}
