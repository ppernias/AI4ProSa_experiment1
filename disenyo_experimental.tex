\documentclass[12pt,a4paper]{article}
\usepackage[spanish]{babel}
\usepackage[utf8]{inputenc}
\usepackage[T1]{fontenc}
\usepackage{geometry}
\usepackage{enumitem}
\usepackage{booktabs}
\usepackage{hyperref}

\geometry{margin=2.5cm}

\title{\textbf{Diseño Experimental -- Primera Iteración}\\
Validación de Tutor IA para Aprendizaje Autónomo\\
Proyecto AI4PROSA}
\date{07-02-2026}

\begin{document}

\maketitle

\section{Pregunta de Investigación Central}

¿Podemos confiar en una IA para tutorizar a nuestros alumnos en la adquisición de conocimientos, reduciendo la carga del docente y aumentando la personalización del aprendizaje?

Esta primera iteración busca validar si un tutor basado en inteligencia artificial puede:

\begin{itemize}
    \item Enseñar eficazmente los contenidos curriculares
    \item Reducir la carga de trabajo del docente durante la sesión
    \item Personalizar el aprendizaje adaptándose a las necesidades individuales
    \item Fomentar la autonomía del alumno, minimizando su dependencia del profesor
\end{itemize}

\section{Objetivo del Experimento}

Validar con un grupo piloto de varios estudiantes (puede variar el número de estudiantes por grupo) el tutor IA desarrollado para la \textbf{Lección 2: ``Obligaciones''} del módulo de seguridad de productos infantiles.

\section{Contexto y Participantes}

\subsection*{Perfil de los estudiantes}

Estudiantes universitarios cursando materias de seguridad de productos infantiles. Deben comprender cómo la legislación europea regula esta área.

\subsection*{Configuración}

\begin{itemize}
    \item $N \geq 3$ estudiantes (mínimo), pudiendo ser superior
    \item 1 profesor/investigador
    \item Laboratorio con control docente
    \item Duración máxima: 2 horas
\end{itemize}

\section{Fases del Experimento}

\subsection{Fase 1: Pre-Experimento}

Preparación técnica y logística previa.

\begin{itemize}
    \item \textbf{Tarea 1: Validación lingüística y pedagógica.} El profesor realizará el proceso poniéndose en el lugar del alumno para verificar traducción y comprensión.
    \item \textbf{Tarea 2: Preparación del espacio físico.} Aula con ordenadores (uno para cada alumno), conexión estable a Internet y acceso a la plataforma.
    \item \textbf{Tarea 3: Registro de participantes.} Enviar a la Universidad de Alicante los datos de los estudiantes y profesor.
    \item \textbf{Tarea 4: Pruebas técnicas previas.} Verificación de acceso a la plataforma y funcionamiento general.
\end{itemize}

\subsection{Fase 2: Experimento}

Sesión de aprendizaje con el tutor IA (máx. 2 horas).

\begin{enumerate}
    \item Reunir a los estudiantes en el espacio preparado.
    \item Explicar funcionamiento del tutor (5--10 minutos).
    \item Comunicar la regla fundamental: El tutor no sustituye al profesor, sino que reduce su carga.
    \item Adoptar rol de observador, sin intervenir académicamente.
    \item Observar y registrar las incidencias.
\end{enumerate}

\subsection{Fase 3: Post-Experimento}

Evaluación inmediata (40 minutos) mediante cuatro instrumentos:

\begin{itemize}
    \item Instrumento 1: Test de contenidos (eficacia)
    \item Instrumento 2: Cuestionario cuantitativo para alumnos
    \item Instrumento 3: Cuestionario cuantitativo para el profesor
    \item Instrumento 4: Evaluación cualitativa grupal
\end{itemize}

\section{Síntesis de Instrumentos}

\begin{table}[h]
\centering
\begin{tabular}{p{3cm} p{3cm} p{3cm} p{2cm}}
\toprule
\textbf{Instrumento} & \textbf{Qué mide} & \textbf{Quién responde} & \textbf{Duración} \\
\midrule
Test de contenidos & Eficacia del aprendizaje & Alumnos (individual) & 15 min \\
Cuestionario alumnos & Personalización y autonomía percibida & Alumnos (individual) & 7 min \\
Cuestionario profesor & Ahorro de tiempo y calidad observada & Profesor (individual) & 7 min \\
Evaluación cualitativa & Experiencia y mejoras & Alumnos (grupal) & 10 min \\
\bottomrule
\end{tabular}
\end{table}

\section{Instrumentos Detallados}

Las secciones completas incluyen los cuestionarios para alumnos y profesor, así como el guion de la evaluación cualitativa grupal. Se mantienen todas las preguntas y escalas Likert en formato enumerado o tipo lista, según corresponda.

\section{Criterios de Éxito Preliminares}

\begin{itemize}
    \item \textbf{Aprendizaje efectivo:} Media del test $\geq 7/10$
    \item \textbf{Reducción de carga docente:} Menos de 10 intervenciones
    \item \textbf{Personalización percibida:} Promedio $\geq 3.5/5$
    \item \textbf{Autonomía lograda:} Promedio $\geq 3.5/5$
    \item \textbf{Viabilidad docente:} Profesor puntúa $\geq 4/5$
\end{itemize}

\section{Contacto y Soporte}

Universidad de Alicante -- Equipo AI4PROSA

\bigskip

\noindent
\textbf{Fecha de última actualización:} 07-02-2026 \\
\textbf{Responsable:} Equipo AI4PROSA -- Universidad de Alicante

\end{document}
