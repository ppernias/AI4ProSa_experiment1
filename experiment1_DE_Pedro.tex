\documentclass[11pt,a4paper]{article}

\usepackage[utf8]{inputenc}
\usepackage[T1]{fontenc}
\usepackage[spanish]{babel}
\usepackage{geometry}
\usepackage{setspace}
\usepackage{booktabs}
\usepackage{longtable}
\usepackage{graphicx}

\geometry{margin=2.5cm}
\onehalfspacing

\title{Validación Experimental de un Tutor Basado en Inteligencia Artificial para el Aprendizaje Autónomo}
\author{Proyecto AI4PROSA \\ Universidad de Alicante}
\date{}

\begin{document}
\maketitle

\section{Introducción}

Este informe presenta los resultados de una primera iteración experimental destinada a validar un tutor basado en inteligencia artificial como apoyo al aprendizaje autónomo en educación superior. El experimento se enmarca en el proyecto AI4PROSA y se centra en la Lección 2 del módulo de seguridad de productos infantiles, relativa a las obligaciones de los agentes económicos en la legislación europea.

El objetivo general del estudio es analizar si un tutor IA puede:
\begin{itemize}
    \item facilitar una adquisición eficaz de contenidos,
    \item fomentar la autonomía del estudiante,
    \item personalizar el aprendizaje,
    \item y reducir la carga docente durante la sesión.
\end{itemize}

El diseño experimental combina instrumentos cuantitativos y cualitativos con un enfoque de métodos mixtos, permitiendo la triangulación de resultados.

\section{Diseño Experimental}

El experimento se llevó a cabo con un grupo piloto de cinco estudiantes universitarios y un docente investigador. La sesión tuvo una duración máxima de dos horas y se desarrolló en un entorno controlado de laboratorio.

\subsection{Instrumentos de Evaluación}

Se emplearon cuatro instrumentos de evaluación, definidos previamente en el diseño experimental:

\begin{enumerate}
    \item Test de evaluación de contenidos (eficacia del aprendizaje).
    \item Cuestionario cuantitativo para estudiantes (personalización y autonomía percibida).
    \item Cuestionario cuantitativo para el docente (carga de trabajo y calidad observada).
    \item Entrevista cualitativa grupal post-experimento.
\end{enumerate}

Se establecieron criterios de éxito preliminares para cada dimensión evaluada, que sirven como referencia para el análisis de resultados.

\section{Resultados}

\subsection{Eficacia del Aprendizaje: Test de Contenidos}

El test de contenidos evaluó la adquisición de los conocimientos curriculares trabajados durante la sesión. El criterio de éxito definido fue una puntuación media igual o superior a 7 sobre 10.

Los cinco participantes completaron el test. Todos declararon un conocimiento previo muy bajo o inexistente sobre la temática antes de la intervención.

\begin{table}[h]
\centering
\begin{tabular}{cc}
\toprule
Estudiante & Puntuación (/10) \\
\midrule
E1 & 8 \\
E2 & 9 \\
E3 & 9 \\
E4 & 9 \\
E5 & 9 \\
\bottomrule
\end{tabular}
\caption{Resultados del test de evaluación de contenidos}
\end{table}

La puntuación media obtenida fue de 8.8 sobre 10, superando ampliamente el criterio de éxito establecido. Estos resultados indican una adquisición eficaz de los contenidos evaluados.

La entrevista cualitativa valida estos resultados al mostrar que los estudiantes comprendieron los conceptos clave y la lógica normativa subyacente, más allá de una simple memorización.

\subsection{Percepción del Estudiante: Personalización y Autonomía}

El cuestionario cuantitativo para estudiantes utilizó una escala Likert de 1 a 5. Los criterios de éxito fueron una media igual o superior a 3.5 tanto en personalización como en autonomía.

\subsubsection{Personalización Percibida}

Los ítems relacionados con la personalización mostraron una valoración positiva general, con una media aproximada de 4.1 sobre 5. Los estudiantes percibieron que el tutor respondía a sus preguntas individuales y permitía avanzar a ritmos distintos.

La entrevista cualitativa matiza estos resultados, indicando que la personalización es principalmente reactiva: el tutor responde adecuadamente a las preguntas formuladas, pero no siempre reconduce de forma proactiva cuando el estudiante se bloquea.

\subsubsection{Autonomía del Aprendizaje}

La autonomía fue la dimensión mejor valorada. Los estudiantes indicaron que podían resolver dudas sin recurrir al docente (5.0/5) y aprender de forma autónoma (4.8/5).

Estos resultados se ven reforzados por la entrevista, en la que los participantes destacan la ausencia de presión, la posibilidad de repreguntar libremente y el control del ritmo de aprendizaje.

\subsubsection{Preferencia y Uso Futuro}

La preferencia frente al modelo tradicional fue moderada (3.6/5), mientras que la intención de reutilizar el tutor en otros temas fue claramente positiva (4.0/5). Los estudiantes conciben el tutor como un complemento al profesor, no como un sustituto.

\subsection{Perspectiva del Docente: Carga de Trabajo y Viabilidad}

El cuestionario del docente evaluó la reducción de carga docente, la calidad pedagógica observada y la viabilidad del tutor.

Durante la sesión:
\begin{itemize}
    \item no se registraron intervenciones académicas del docente,
    \item ningún estudiante requirió apoyo individual.
\end{itemize}

Estos resultados cumplen ampliamente el criterio de éxito establecido. El docente valoró la reducción de carga de trabajo con un 4 sobre 5.

En cuanto a la calidad pedagógica, las explicaciones del tutor fueron valoradas como adecuadas pero mejorables (3/5), señalando la necesidad ocasional de complementar o corregir contenidos.

A pesar de ello, el tutor fue considerado viable para su uso regular (4/5) y altamente recomendable para otros docentes (5/5).

\section{Discusión}

La integración de los resultados cuantitativos y cualitativos muestra una alta coherencia entre instrumentos. El tutor IA demuestra una elevada eficacia para la adquisición de contenidos, especialmente en estudiantes con conocimiento previo limitado.

La autonomía del estudiante emerge como un factor clave que media la eficacia del aprendizaje y contribuye directamente a la reducción de la carga docente. La percepción de personalización es positiva, aunque limitada por la falta de andamiaje pedagógico proactivo.

Tanto estudiantes como docente coinciden en señalar áreas de mejora, especialmente en la estructuración de las explicaciones, el soporte visual y la capacidad del tutor para guiar al estudiante cuando no sabe cómo avanzar.

Estas limitaciones no invalidan los resultados obtenidos, sino que delimitan el alcance actual del sistema y proporcionan criterios claros para futuras iteraciones.

\section{Conclusiones}

Los resultados de esta primera iteración experimental indican que el tutor basado en inteligencia artificial:
\begin{itemize}
    \item facilita una adquisición eficaz de contenidos,
    \item fomenta una alta autonomía del estudiante,
    \item reduce significativamente la carga docente durante la sesión,
    \item y es considerado viable y recomendable desde la perspectiva profesional.
\end{itemize}

El tutor se valida como una herramienta de apoyo al aprendizaje, coherente con el diseño experimental planteado. Las mejoras identificadas constituyen una base sólida para el desarrollo de futuras versiones del sistema.

\end{document}

