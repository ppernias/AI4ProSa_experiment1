Diseño Experimental - Primera Iteración
Validación de Tutor IA para Aprendizaje Autónomo
Proyecto AI4PROSA

1.Pregunta de Investigación Central
¿Podemos confiar en una IA para tutorizar a nuestros alumnos en la adquisición de conocimientos, reduciendo la carga del docente y aumentando la
personalización del aprendizaje?

Esta primera iteración busca validar si un tutor basado en inteligencia artificial puede:
- Enseñar eficazmente los contenidos curriculares
- Reducir la carga de trabajo del docente durante la sesión
- Personalizar el aprendizaje adaptándose a las necesidades individuales
- Fomentar la autonomía del alumno, minimizando su dependencia del profesor

2. Objetivo del Experimento
Validar con un grupo piloto de varios estudiantes (puede variar el número de estudiantes por grupo) el tutor IA desarrollado para la Lección 2: “Obligaciones” del módulo de seguridad de productos infantiles.

3. Contexto y Participantes
Perfil de los estudiantes
Estudiantes universitarios cursando materias de seguridad de productos infantiles. Deben comprender cómo la legislación europea regula esta área.

Configuración
N = 3 estudiantes mínimo, aunque puede ser superior
1 profesor/investigador
Laboratorio con control docente
Duración máxima: 2 horas

4. Fases del Experimento
4.1. Fase 1: Pre-Experimento
Preparación técnica y logística previa.

Tarea 1: Validación lingüística y pedagógica. El profesor realizará el proceso poniéndose en el lugar del alumno para verificar traducción y comprensión.

Tarea 2: Preparación del espacio físico Aula con ordenadores (uno para acada
alumno), conexión estable a Internet y acceso a la plataforma.

Tarea 3: Registro de participantes Enviar a la Universidad de Alicante los datos de los estudiantes y profesor.

Tarea 4: Pruebas técnicas previas Verificación de acceso a la plataforma y funcionamiento general.

4.2. Fase 2: Experimento
Sesión de aprendizaje con el tutor IA (máx. 2 horas).
1. Reunir a los estudiantes en el espacio preparado.
2. Explicar funcionamiento del tutor (5-10 minutos).
3. Comunicar la regla fundamental: El tutor no sustituye al profesor, sino que reduce su carga.
4. Adoptar rol de observador, sin intervenir académicamente.
5. Observar y registrar las incidencias.

4.3. Fase 3: Post-Experimento
Evaluación inmediata (40 minutos) mediante cuatro instrumentos:
Instrumento 1: Test de contenidos (eficacia)
Instrumento 2: Cuestionario cuantitativo para alumnos
Instrumento 3: Cuestionario cuantitativo para el profesor
Instrumento 4: Evaluación cualitativa grupal

5. Síntesis de Instrumentos

Listado de instrumentos utilizados en formato csv (archivo separado por comas):
Instrumento,Qué mide,Quién responde,Duración,URL
Test de contenidos, Eficacia del aprendizaje, Alumnos (individual), 15min
Cuestionario alumnos, Personalización y autonomía percibida, Alumnos(individual),7 min
Cuestionario profesor Ahorro de tiempo y calidad observada, Profesor(individual),7min
Evaluación cualitativa,Experiencia y mejoras, Alumnos (grupal),10 min

6. Instrumentos Detallados
Las secciones completas incluyen los cuestionarios para alumnos y profesor, así como el guion de la evaluación cualitativa grupal. (Se mantienen todas las preguntas y escalas Likert en formato enumerado o tipo lista, según corresponda).

7. Criterios de Éxito Preliminares

- Aprendizaje efectivo: Media del test ≥ 7/10
- Reducción de carga docente: Menos de 10 intervenciones
- Personalización percibida: Promedio ≥ 3.5/5
- Autonomía lograda: Promedio ≥ 3.5/5
- Viabilidad docente: Profesor puntúa ≥ 4/5

8. Contacto y Soporte
Universidad de Alicante – Equipo AI4PROSA

Fecha de última actualización: 07-02-2026
Responsable: Equipo AI4PROSA - Universidad de Alicante
