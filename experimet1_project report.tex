\documentclass[11pt,a4paper]{article}

\usepackage[utf8]{inputenc}
\usepackage[T1]{fontenc}
\usepackage{lmodern}
\usepackage{geometry}
\usepackage{setspace}
\usepackage{enumitem}

\geometry{margin=2.5cm}
\onehalfspacing

\title{Summary of Experiment 1 Evaluation}
\date{2026-02-07}

\begin{document}
\maketitle

\section{Purpose of the Evaluation}

This section summarises the evaluation results of Experiment 1, which aimed to assess the viability of an AI-based tutor for supporting autonomous learning, reducing teaching workload, and enhancing personalisation in higher education contexts.  
The evaluation integrates four instruments: a content test, student quantitative questionnaires, a teacher quantitative questionnaire, and qualitative group interviews.

\section{Learning Effectiveness}

Results from the content test indicate that the AI tutor is capable of effectively supporting knowledge acquisition. The predefined success criterion (mean score $\geq 7/10$) was met in the majority of student groups.

Performance, however, was not homogeneous across contexts. Variations were observed depending on language, test format, and cognitive load. These findings suggest that learning effectiveness is sensitive to pedagogical and linguistic adaptation.

\section{Personalisation and Student Autonomy}

Quantitative and qualitative data show that students generally perceive a meaningful level of personalisation, particularly in terms of responding to individual questions and allowing self-paced learning.

The AI tutor supports student autonomy by enabling learners to progress independently and reduce immediate reliance on the teacher. Nevertheless, autonomy is limited when the system does not adequately adapt explanations to the learner’s initial level or provide alternative strategies when learning difficulties arise.

\section{Teaching Workload Reduction}

From the teacher’s perspective, the AI tutor contributes to a reduction in teaching workload by decreasing the number of required interventions and supporting student self-regulation.

This effect is clearly observed in standard learning contexts, while in more demanding or complex scenarios the reduction in workload is less consistent. Overall, the results support the tutor’s role as an effective teaching aid rather than a replacement for the teacher.

\section{User Experience and Qualitative Insights}

Qualitative interviews highlight that students perceive the AI tutor as more interactive than traditional lecture-based instruction. However, usability limitations were identified, particularly related to long text-based responses and limited multimodal support.

Students expressed the need for improved visual structuring, reduced textual density, and the integration of multimodal elements (e.g. visual summaries, audio support). These issues are pedagogical and interface-related rather than content-related.

\section{Overall Conclusions}

The evaluation confirms that the AI tutor is a viable tool for supporting autonomous learning and reducing teaching workload, largely fulfilling the objectives defined in the experimental design.

The results justify proceeding to a subsequent iteration, focusing on:
\begin{itemize}
  \item Improved adaptation to learners’ initial knowledge levels,
  \item Enhanced pedagogical design,
  \item Reduced cognitive load through better visual and multimodal support.
\end{itemize}

These improvements are expected to increase consistency of outcomes across languages and learning contexts and to strengthen the tutor’s applicability in real educational settings.

\end{document}
